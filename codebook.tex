% Options for packages loaded elsewhere
\PassOptionsToPackage{unicode}{hyperref}
\PassOptionsToPackage{hyphens}{url}
%
\documentclass[
]{article}
\usepackage{lmodern}
\usepackage{amssymb,amsmath}
\usepackage{ifxetex,ifluatex}
\ifnum 0\ifxetex 1\fi\ifluatex 1\fi=0 % if pdftex
  \usepackage[T1]{fontenc}
  \usepackage[utf8]{inputenc}
  \usepackage{textcomp} % provide euro and other symbols
\else % if luatex or xetex
  \usepackage{unicode-math}
  \defaultfontfeatures{Scale=MatchLowercase}
  \defaultfontfeatures[\rmfamily]{Ligatures=TeX,Scale=1}
\fi
% Use upquote if available, for straight quotes in verbatim environments
\IfFileExists{upquote.sty}{\usepackage{upquote}}{}
\IfFileExists{microtype.sty}{% use microtype if available
  \usepackage[]{microtype}
  \UseMicrotypeSet[protrusion]{basicmath} % disable protrusion for tt fonts
}{}
\makeatletter
\@ifundefined{KOMAClassName}{% if non-KOMA class
  \IfFileExists{parskip.sty}{%
    \usepackage{parskip}
  }{% else
    \setlength{\parindent}{0pt}
    \setlength{\parskip}{6pt plus 2pt minus 1pt}}
}{% if KOMA class
  \KOMAoptions{parskip=half}}
\makeatother
\usepackage{xcolor}
\IfFileExists{xurl.sty}{\usepackage{xurl}}{} % add URL line breaks if available
\IfFileExists{bookmark.sty}{\usepackage{bookmark}}{\usepackage{hyperref}}
\hypersetup{
  pdftitle={Codebook for CANDOUR Conjoint Analysis},
  pdfauthor={Raymond Duch et al},
  hidelinks,
  pdfcreator={LaTeX via pandoc}}
\urlstyle{same} % disable monospaced font for URLs
\usepackage[margin=1in]{geometry}
\usepackage{longtable,booktabs}
% Correct order of tables after \paragraph or \subparagraph
\usepackage{etoolbox}
\makeatletter
\patchcmd\longtable{\par}{\if@noskipsec\mbox{}\fi\par}{}{}
\makeatother
% Allow footnotes in longtable head/foot
\IfFileExists{footnotehyper.sty}{\usepackage{footnotehyper}}{\usepackage{footnote}}
\makesavenoteenv{longtable}
\usepackage{graphicx,grffile}
\makeatletter
\def\maxwidth{\ifdim\Gin@nat@width>\linewidth\linewidth\else\Gin@nat@width\fi}
\def\maxheight{\ifdim\Gin@nat@height>\textheight\textheight\else\Gin@nat@height\fi}
\makeatother
% Scale images if necessary, so that they will not overflow the page
% margins by default, and it is still possible to overwrite the defaults
% using explicit options in \includegraphics[width, height, ...]{}
\setkeys{Gin}{width=\maxwidth,height=\maxheight,keepaspectratio}
% Set default figure placement to htbp
\makeatletter
\def\fps@figure{htbp}
\makeatother
\setlength{\emergencystretch}{3em} % prevent overfull lines
\providecommand{\tightlist}{%
  \setlength{\itemsep}{0pt}\setlength{\parskip}{0pt}}
\setcounter{secnumdepth}{-\maxdimen} % remove section numbering

\title{Codebook for CANDOUR Conjoint Analysis}
\author{Raymond Duch et al}
\date{19/12/2020}

\begin{document}
\maketitle

All data used for the conjoint analysis and variables is stored within
the \texttt{data/nbh\_clean\_global.*} files. We provide a raw
\texttt{.CSV} version and a formatted \texttt{.Rds} file which contains
the same data but with factor-variable formatting pre-applied to the
conjoint attribute variables.

Each row of the dataset is a hypothetical vaccine recipient, or
``profile''. Two profiles comprise a single conjoint ``round'', and
forms the basis of the dichotomous choice the respondent makes between
potential recipients. Each respondent is presented with 8 rounds of of
the conjoint experiment, and therefore makes 8 choices.

\hypertarget{respondent-characteristics}{%
\subsection{Respondent
Characteristics}\label{respondent-characteristics}}

\textbf{id} -- \emph{Character variable} -- Distinguishes individual
respondents in the dataset, and takes the form of
\texttt{{[}country{]}\_} and a unique number corresponding to each
respondent within that country.

\begin{longtable}[]{@{}lr@{}}
\toprule
& id\tabularnewline
\midrule
\endhead
Obs. & 214960\tabularnewline
Unique & 13435\tabularnewline
Missing & 0\tabularnewline
\bottomrule
\end{longtable}

E.g.,

\begin{verbatim}
## [1] "Canada_1, Canada_2, Canada_3, Canada_4, Canada_5, ..."
\end{verbatim}

\textbf{country} -- Country location of respondent.

\begin{longtable}[]{@{}lr@{}}
\toprule
Value & Obs.\tabularnewline
\midrule
\endhead
Australia & 21600\tabularnewline
Brazil & 22784\tabularnewline
Canada & 18400\tabularnewline
Chile & 17088\tabularnewline
China & 20656\tabularnewline
Colombia & 13728\tabularnewline
France & 16496\tabularnewline
India & 10864\tabularnewline
Italy & 16976\tabularnewline
Spain & 18448\tabularnewline
Uganda & 2688\tabularnewline
UK & 16912\tabularnewline
US & 18320\tabularnewline
Missing & 0\tabularnewline
\bottomrule
\end{longtable}

\newpage

\textbf{age} -- Age of respondent.

\begin{longtable}[]{@{}ll@{}}
\toprule
& age\tabularnewline
\midrule
\endhead
Obs. & 214960\tabularnewline
Mean & 44.7\tabularnewline
Std.Dev. & 47.4\tabularnewline
Min. & 2\tabularnewline
Max. & 1987\tabularnewline
Missing & 0\tabularnewline
\bottomrule
\end{longtable}

\textbf{gender} -- Gender of respondent.

\begin{longtable}[]{@{}lr@{}}
\toprule
Value & Obs.\tabularnewline
\midrule
\endhead
Female & 107136\tabularnewline
Male & 107024\tabularnewline
Other & 272\tabularnewline
Missing & 528\tabularnewline
\bottomrule
\end{longtable}

\textbf{ideology} -- Ideological self-placement of respondent on an
11-point scale from 0 (Left) to 10 (Right). Note, this question was not
asked in China.

\begin{longtable}[]{@{}ll@{}}
\toprule
& ideology\tabularnewline
\midrule
\endhead
Obs. & 214960\tabularnewline
Mean & 5.2\tabularnewline
Std.Dev. & 2.6\tabularnewline
Min. & 0\tabularnewline
Max. & 10\tabularnewline
Missing & 50592\tabularnewline
\bottomrule
\end{longtable}

\textbf{ind\_inc} -- Dichotomous classification of respondents' income
level. \texttt{High} and \texttt{Low} values are relative to the median
estimated income for each country.

\begin{longtable}[]{@{}lr@{}}
\toprule
Value & Obs.\tabularnewline
\midrule
\endhead
High & 85440\tabularnewline
Low & 106176\tabularnewline
Missing & 23344\tabularnewline
\bottomrule
\end{longtable}

\newpage

\textbf{education} -- Trichotomous classification of respondents' level
of education, relative to the educational levels within each country.

\begin{longtable}[]{@{}lr@{}}
\toprule
Value & Obs.\tabularnewline
\midrule
\endhead
High & 98816\tabularnewline
Medium & 75616\tabularnewline
Low & 30224\tabularnewline
Missing & 10304\tabularnewline
\bottomrule
\end{longtable}

\textbf{hes\_covid\_2} -- Five-point Likert scale from
\texttt{Strongly\ agree} to \texttt{Strongly\ disagree}, asking
respondents' opinions on the following statement: ``I am concerned about
serious side effects of the COVID-19 vaccine''.

\begin{longtable}[]{@{}lr@{}}
\toprule
Value & Obs.\tabularnewline
\midrule
\endhead
Strongly agree & 64400\tabularnewline
Agree & 75312\tabularnewline
Disagree & 21504\tabularnewline
Neither agree nor disagree & 39632\tabularnewline
Strongly disagree & 8864\tabularnewline
Do not know & 5040\tabularnewline
Missing & 208\tabularnewline
\bottomrule
\end{longtable}

\textbf{wtp\_access} -- Response to the following prompt:

"Talking about vaccines in general, in some countries vaccines are only
available from the government either at low or no cost. In some
countries vaccines are only available for private purchase. And in some
countries vaccines are available from the government but citizens can
pay privately to gain early access.

Which of these three approaches do you think should be applied to the
COVID-19 vaccine? Would you prefer

\begin{itemize}
\tightlist
\item
  Vaccines only made available by government at low or no cost?
\item
  Vaccines are only available for private purchase?
\item
  Vaccines made available by government but citizens can pay privately
  to gain access?"
\end{itemize}

\begin{longtable}[]{@{}lr@{}}
\toprule
\begin{minipage}[b]{0.87\columnwidth}\raggedright
Value\strut
\end{minipage} & \begin{minipage}[b]{0.07\columnwidth}\raggedleft
Obs.\strut
\end{minipage}\tabularnewline
\midrule
\endhead
\begin{minipage}[t]{0.87\columnwidth}\raggedright
Do not know\strut
\end{minipage} & \begin{minipage}[t]{0.07\columnwidth}\raggedleft
14448\strut
\end{minipage}\tabularnewline
\begin{minipage}[t]{0.87\columnwidth}\raggedright
Vaccines only made available by government at low or no cost\strut
\end{minipage} & \begin{minipage}[t]{0.07\columnwidth}\raggedleft
155568\strut
\end{minipage}\tabularnewline
\begin{minipage}[t]{0.87\columnwidth}\raggedright
Vaccines made available by government but citizens can pay privately to
gain access\strut
\end{minipage} & \begin{minipage}[t]{0.07\columnwidth}\raggedleft
38608\strut
\end{minipage}\tabularnewline
\begin{minipage}[t]{0.87\columnwidth}\raggedright
Vaccines are only available for private purchase\strut
\end{minipage} & \begin{minipage}[t]{0.07\columnwidth}\raggedleft
6128\strut
\end{minipage}\tabularnewline
\begin{minipage}[t]{0.87\columnwidth}\raggedright
Prefer not to say\strut
\end{minipage} & \begin{minipage}[t]{0.07\columnwidth}\raggedleft
208\strut
\end{minipage}\tabularnewline
\begin{minipage}[t]{0.87\columnwidth}\raggedright
Missing\strut
\end{minipage} & \begin{minipage}[t]{0.07\columnwidth}\raggedleft
0\strut
\end{minipage}\tabularnewline
\bottomrule
\end{longtable}

\newpage

\textbf{wtp\_private} -- Response to the following prompt:

``Consider the following situation: a COVID-19 vaccine becomes available
and is provided by government health agencies. For 80 out of 100 people
the vaccine would provide protection for at least 18 months. But there
are limited initial supplies of the vaccine. For this reason, you would
have to wait 6 months before you could receive it. If a COVID-19 vaccine
was also available for private purchase and you could receive it
immediately would you considering buying it?''

\begin{longtable}[]{@{}lr@{}}
\toprule
Value & Obs.\tabularnewline
\midrule
\endhead
Do not know & 47440\tabularnewline
No & 70224\tabularnewline
Yes & 97152\tabularnewline
Prefer not to say & 144\tabularnewline
Missing & 0\tabularnewline
\bottomrule
\end{longtable}

\textbf{int\_pol\_implem\_6} -- 0-100 scale, asking respondent's how
much they agree with the following statement: ``The government should
make the COVID-19 vaccine mandatory for everybody''

\begin{longtable}[]{@{}ll@{}}
\toprule
& int\_pol\_implem\_6\tabularnewline
\midrule
\endhead
Obs. & 214960\tabularnewline
Mean & 57.5\tabularnewline
Std.Dev. & 37.5\tabularnewline
Min. & 0\tabularnewline
Max. & 100\tabularnewline
Missing & 14720\tabularnewline
\bottomrule
\end{longtable}

\newpage

\hypertarget{conjoint-variables}{%
\subsection{Conjoint Variables}\label{conjoint-variables}}

\hypertarget{meta-data}{%
\subsubsection{Meta-data}\label{meta-data}}

\textbf{person} - distinguishes rounds of the conjoint experiment, and
takes the form of \texttt{person{[}i{]}} where \texttt{i} corresponds to
the ith comparison between two hypothetical profiles. Each respondent
was presented with 8 rounds of the conjoint.

\begin{longtable}[]{@{}lr@{}}
\toprule
& person\tabularnewline
\midrule
\endhead
Obs. & 214960\tabularnewline
Unique & 8\tabularnewline
Missing & 0\tabularnewline
\bottomrule
\end{longtable}

\textbf{candidate} -- distinguishes the two candidate profiles within
each round of the conjoint experiment (``A'' or ``B'').

\begin{longtable}[]{@{}lr@{}}
\toprule
& candidate\tabularnewline
\midrule
\endhead
Obs. & 214960\tabularnewline
Unique & 2\tabularnewline
Missing & 0\tabularnewline
\bottomrule
\end{longtable}

\textbf{ans} -- indicates which candidate (``A'' or ``B'') the
respondent chose in the round of the experiment corresponding to this
observation.

\begin{longtable}[]{@{}lr@{}}
\toprule
& ans\tabularnewline
\midrule
\endhead
Obs. & 214960\tabularnewline
Unique & 2\tabularnewline
Missing & 0\tabularnewline
\bottomrule
\end{longtable}

\textbf{select} -- binary indicator of whether each profile (the unit of
observation) was chosen (1) or not (0).

\begin{longtable}[]{@{}lr@{}}
\toprule
Value & Obs.\tabularnewline
\midrule
\endhead
1 & 107480\tabularnewline
0 & 107480\tabularnewline
Missing & 0\tabularnewline
\bottomrule
\end{longtable}

\newpage

\hypertarget{attributes}{%
\subsubsection{Attributes}\label{attributes}}

\emph{Note: reference levels for each attribute are indicated with an
asterisk.}

\textbf{vulnerability} -- Risk of COVID-19 related death.

\begin{longtable}[]{@{}lr@{}}
\toprule
Value & Obs.\tabularnewline
\midrule
\endhead
*Average risk of COVID-19 death & 71190\tabularnewline
Moderate (Twice the average risk of COVID-19 death) &
71839\tabularnewline
High (Five times the average risk of COVID-19 death) &
71931\tabularnewline
Missing & 0\tabularnewline
\bottomrule
\end{longtable}

\textbf{transmission} -- Risk of catching and transmitting the COVID-19
virus.

\begin{longtable}[]{@{}lr@{}}
\toprule
\begin{minipage}[b]{0.88\columnwidth}\raggedright
Value\strut
\end{minipage} & \begin{minipage}[b]{0.06\columnwidth}\raggedleft
Obs.\strut
\end{minipage}\tabularnewline
\midrule
\endhead
\begin{minipage}[t]{0.88\columnwidth}\raggedright
*Average risk of catching and transmitting the COVID-19 virus\strut
\end{minipage} & \begin{minipage}[t]{0.06\columnwidth}\raggedleft
71593\strut
\end{minipage}\tabularnewline
\begin{minipage}[t]{0.88\columnwidth}\raggedright
Moderate risk (Twice the average risk of catching and transmitting the
COVID-19 virus)\strut
\end{minipage} & \begin{minipage}[t]{0.06\columnwidth}\raggedleft
71765\strut
\end{minipage}\tabularnewline
\begin{minipage}[t]{0.88\columnwidth}\raggedright
High risk (Five times the average risk of catching and transmitting the
COVID-19 virus)\strut
\end{minipage} & \begin{minipage}[t]{0.06\columnwidth}\raggedleft
71602\strut
\end{minipage}\tabularnewline
\begin{minipage}[t]{0.88\columnwidth}\raggedright
Missing\strut
\end{minipage} & \begin{minipage}[t]{0.06\columnwidth}\raggedleft
0\strut
\end{minipage}\tabularnewline
\bottomrule
\end{longtable}

\textbf{income} -- Income level.

\begin{longtable}[]{@{}lr@{}}
\toprule
Value & Obs.\tabularnewline
\midrule
\endhead
*Lowest 20\% income level & 71439\tabularnewline
Average income level & 71774\tabularnewline
Highest 20\% income level & 71747\tabularnewline
Missing & 0\tabularnewline
\bottomrule
\end{longtable}

\textbf{occupation} -- Occupation status.

\begin{longtable}[]{@{}lr@{}}
\toprule
Value & Obs.\tabularnewline
\midrule
\endhead
*Not working & 27068\tabularnewline
Non-Key worker: Can work at home & 26807\tabularnewline
Non-Key worker: Cannot work at home & 26919\tabularnewline
Key worker: Education and childcare & 26738\tabularnewline
Key worker: Factory worker & 26692\tabularnewline
Key worker: Water and electricity service & 26920\tabularnewline
Key worker: Police and fire-fighting & 26819\tabularnewline
Key worker: Health and social care & 26997\tabularnewline
Missing & 0\tabularnewline
\bottomrule
\end{longtable}

\newpage

\textbf{age\_category} -- Age category.

\begin{longtable}[]{@{}lr@{}}
\toprule
Value & Obs.\tabularnewline
\midrule
\endhead
*25 years old & 53900\tabularnewline
40 years old & 53715\tabularnewline
65 years old & 53676\tabularnewline
79 years old & 53669\tabularnewline
Missing & 0\tabularnewline
\bottomrule
\end{longtable}

\end{document}
